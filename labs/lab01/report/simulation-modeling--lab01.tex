% Options for packages loaded elsewhere
% Options for packages loaded elsewhere
\PassOptionsToPackage{unicode}{hyperref}
\PassOptionsToPackage{hyphens}{url}
\PassOptionsToPackage{dvipsnames,svgnames,x11names}{xcolor}
%
\documentclass[
  letterpaper,
  DIV=11,
  numbers=noendperiod]{scrartcl}
\usepackage{xcolor}
\usepackage{amsmath,amssymb}
\setcounter{secnumdepth}{5}
\usepackage{iftex}
\ifPDFTeX
  \usepackage[T1]{fontenc}
  \usepackage[utf8]{inputenc}
  \usepackage{textcomp} % provide euro and other symbols
\else % if luatex or xetex
  \usepackage{unicode-math} % this also loads fontspec
  \defaultfontfeatures{Scale=MatchLowercase}
  \defaultfontfeatures[\rmfamily]{Ligatures=TeX,Scale=1}
\fi
\usepackage{lmodern}
\ifPDFTeX\else
  % xetex/luatex font selection
\fi
% Use upquote if available, for straight quotes in verbatim environments
\IfFileExists{upquote.sty}{\usepackage{upquote}}{}
\IfFileExists{microtype.sty}{% use microtype if available
  \usepackage[]{microtype}
  \UseMicrotypeSet[protrusion]{basicmath} % disable protrusion for tt fonts
}{}
\makeatletter
\@ifundefined{KOMAClassName}{% if non-KOMA class
  \IfFileExists{parskip.sty}{%
    \usepackage{parskip}
  }{% else
    \setlength{\parindent}{0pt}
    \setlength{\parskip}{6pt plus 2pt minus 1pt}}
}{% if KOMA class
  \KOMAoptions{parskip=half}}
\makeatother
% Make \paragraph and \subparagraph free-standing
\makeatletter
\ifx\paragraph\undefined\else
  \let\oldparagraph\paragraph
  \renewcommand{\paragraph}{
    \@ifstar
      \xxxParagraphStar
      \xxxParagraphNoStar
  }
  \newcommand{\xxxParagraphStar}[1]{\oldparagraph*{#1}\mbox{}}
  \newcommand{\xxxParagraphNoStar}[1]{\oldparagraph{#1}\mbox{}}
\fi
\ifx\subparagraph\undefined\else
  \let\oldsubparagraph\subparagraph
  \renewcommand{\subparagraph}{
    \@ifstar
      \xxxSubParagraphStar
      \xxxSubParagraphNoStar
  }
  \newcommand{\xxxSubParagraphStar}[1]{\oldsubparagraph*{#1}\mbox{}}
  \newcommand{\xxxSubParagraphNoStar}[1]{\oldsubparagraph{#1}\mbox{}}
\fi
\makeatother


\usepackage{longtable,booktabs,array}
\usepackage{calc} % for calculating minipage widths
% Correct order of tables after \paragraph or \subparagraph
\usepackage{etoolbox}
\makeatletter
\patchcmd\longtable{\par}{\if@noskipsec\mbox{}\fi\par}{}{}
\makeatother
% Allow footnotes in longtable head/foot
\IfFileExists{footnotehyper.sty}{\usepackage{footnotehyper}}{\usepackage{footnote}}
\makesavenoteenv{longtable}
\usepackage{graphicx}
\makeatletter
\newsavebox\pandoc@box
\newcommand*\pandocbounded[1]{% scales image to fit in text height/width
  \sbox\pandoc@box{#1}%
  \Gscale@div\@tempa{\textheight}{\dimexpr\ht\pandoc@box+\dp\pandoc@box\relax}%
  \Gscale@div\@tempb{\linewidth}{\wd\pandoc@box}%
  \ifdim\@tempb\p@<\@tempa\p@\let\@tempa\@tempb\fi% select the smaller of both
  \ifdim\@tempa\p@<\p@\scalebox{\@tempa}{\usebox\pandoc@box}%
  \else\usebox{\pandoc@box}%
  \fi%
}
% Set default figure placement to htbp
\def\fps@figure{htbp}
\makeatother





\setlength{\emergencystretch}{3em} % prevent overfull lines

\providecommand{\tightlist}{%
  \setlength{\itemsep}{0pt}\setlength{\parskip}{0pt}}



 


\KOMAoption{captions}{tableheading}
\makeatletter
\@ifpackageloaded{caption}{}{\usepackage{caption}}
\AtBeginDocument{%
\ifdefined\contentsname
  \renewcommand*\contentsname{Table of contents}
\else
  \newcommand\contentsname{Table of contents}
\fi
\ifdefined\listfigurename
  \renewcommand*\listfigurename{List of Figures}
\else
  \newcommand\listfigurename{List of Figures}
\fi
\ifdefined\listtablename
  \renewcommand*\listtablename{List of Tables}
\else
  \newcommand\listtablename{List of Tables}
\fi
\ifdefined\figurename
  \renewcommand*\figurename{Figure}
\else
  \newcommand\figurename{Figure}
\fi
\ifdefined\tablename
  \renewcommand*\tablename{Table}
\else
  \newcommand\tablename{Table}
\fi
}
\@ifpackageloaded{float}{}{\usepackage{float}}
\floatstyle{ruled}
\@ifundefined{c@chapter}{\newfloat{codelisting}{h}{lop}}{\newfloat{codelisting}{h}{lop}[chapter]}
\floatname{codelisting}{Listing}
\newcommand*\listoflistings{\listof{codelisting}{List of Listings}}
\makeatother
\makeatletter
\makeatother
\makeatletter
\@ifpackageloaded{caption}{}{\usepackage{caption}}
\@ifpackageloaded{subcaption}{}{\usepackage{subcaption}}
\makeatother
\usepackage{bookmark}
\IfFileExists{xurl.sty}{\usepackage{xurl}}{} % add URL line breaks if available
\urlstyle{same}
\hypersetup{
  pdftitle={Лабораторная работа №1},
  pdfauthor={Мария Чувакина},
  colorlinks=true,
  linkcolor={blue},
  filecolor={Maroon},
  citecolor={Blue},
  urlcolor={Blue},
  pdfcreator={LaTeX via pandoc}}


\title{Лабораторная работа №1}
\usepackage{etoolbox}
\makeatletter
\providecommand{\subtitle}[1]{% add subtitle to \maketitle
  \apptocmd{\@title}{\par {\large #1 \par}}{}{}
}
\makeatother
\subtitle{Модель экспоненциального роста}
\author{Мария Чувакина}
\date{Invalid Date}
\begin{document}
\maketitle

\renewcommand*\contentsname{Table of contents}
{
\hypersetup{linkcolor=}
\setcounter{tocdepth}{2}
\tableofcontents
}

\section{Введение}\label{ux432ux432ux435ux434ux435ux43dux438ux435}

Экспоненциальный рост --- это процесс увеличения величины, при котором
скорость роста пропорциональна текущему значению. Модель описывается
дифференциальным уравнением:

\[ \frac{du}{dt} = \alpha u, \quad u(0) = u_0 \]

где: - \(u\) --- текущее значение величины, - \(t\) --- время, -
\(\alpha\) --- константа скорости роста.

\section{Реализация
модели}\label{ux440ux435ux430ux43bux438ux437ux430ux446ux438ux44f-ux43cux43eux434ux435ux43bux438}

\subsection{Подключение
пакетов}\label{ux43fux43eux434ux43aux43bux44eux447ux435ux43dux438ux435-ux43fux430ux43aux435ux442ux43eux432}

```julia using DrWatson @quickactivate ``project'' using
DifferentialEquations using Plots using DataFrames using JLD2

\section{Определение функции
модели}\label{ux43eux43fux440ux435ux434ux435ux43bux435ux43dux438ux435-ux444ux443ux43dux43aux446ux438ux438-ux43cux43eux434ux435ux43bux438}

function exponential\_growth!(du, u, p, t) α = p du{[}1{]} = α *
u{[}1{]} end

\section{Запуск с параметрами по
умолчанию}\label{ux437ux430ux43fux443ux441ux43a-ux441-ux43fux430ux440ux430ux43cux435ux442ux440ux430ux43cux438-ux43fux43e-ux443ux43cux43eux43bux447ux430ux43dux438ux44e}

u0 = {[}1.0{]} α = 0.3 tspan = (0.0, 10.0)

prob = ODEProblem(exponential\_growth!, u0, tspan, α) sol = solve(prob,
Tsit5(), saveat=0.1)

\section{Результаты
моделирования}\label{ux440ux435ux437ux443ux43bux44cux442ux430ux442ux44b-ux43cux43eux434ux435ux43bux438ux440ux43eux432ux430ux43dux438ux44f}

df = DataFrame(t=sol.t, u=first.(sol.u)) first(df, 5) doubling\_time =
log(2) / α println(``Время удвоения:'', round(doubling\_time, digits=2))

\section{Набор
параметров}\label{ux43dux430ux431ux43eux440-ux43fux430ux440ux430ux43cux435ux442ux440ux43eux432}

alpha\_values = {[}0.1, 0.3, 0.5, 0.8, 1.0{]} results = {[}{]}

for α in alpha\_values prob = ODEProblem(exponential\_growth!,
{[}1.0{]}, (0.0, 10.0), α) sol = solve(prob, Tsit5(), saveat=0.1)

\begin{verbatim}
final_pop = last(sol.u)[1]
doubling = log(2) / α

push!(results, (α=α, final_population=final_pop, doubling_time=doubling))
\end{verbatim}

end

results\_df = DataFrame(results) results\_df

\subsection{Графики}\label{ux433ux440ux430ux444ux438ux43aux438}

\begin{figure}[H]

{\centering \pandocbounded{\includegraphics[keepaspectratio]{plots/01_exponential_growth/exponential_growth_α=0.3.png}}

}

\caption{Экспоненциальный рост при α=0.3}

\end{figure}%

\begin{figure}[H]

{\centering \pandocbounded{\includegraphics[keepaspectratio]{plots/02_exponential_growth/parametric_scan_comparison.png}}

}

\caption{Сравнение траекторий}

\end{figure}%

\begin{figure}[H]

{\centering \pandocbounded{\includegraphics[keepaspectratio]{plots/02_exponential_growth/doubling_time_vs_alpha.png}}

}

\caption{Зависимость времени удвоения}

\end{figure}%

\section{Заключение}\label{ux437ux430ux43aux43bux44eux447ux435ux43dux438ux435}

В ходе лабораторной работы была реализована модель экспоненциального
роста. Проведено параметрическое исследование, подтверждающее
теоретические зависимости. Все результаты сохранены, построены графики.




\end{document}
